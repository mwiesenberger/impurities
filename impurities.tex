%%%%%%%%%%%%%%%%%%%%%definitions%%%%%%%%%%%%%%%%%%%%%%%%%%%%%%%%%%%%%%%

\input{../common/header.tex}
\input{../common/newcommands.tex}
\usepackage{minted}

%%%%%%%%%%%%%%%%%%%%%%%%%%%%%DOCUMENT%%%%%%%%%%%%%%%%%%%%%%%%%%%%%%%%%%%%%%%
\begin{document}

\title{The toefl project}
\author{ M.~Wiesenberger and M.~Held}
\maketitle

\begin{abstract}
  This is a program for 2d isothermal blob simulations with multiple ion species
\end{abstract}

\section{Equations}
Currently we implemented $5$ slightly different sets of equations. $n$ is the electron density, $N$ is the ion gyrocentre density and $\rho$
the vorticity density. $\phi$ is the electric potential. We
use Cartesian coordinates $x$, $y$.
\subsection{Model}

The resulting dimensionless equations for the evolution of the gyro-center
densities $n_s({\bf x},t)$ for each species are:

\begin{equation}
    \partial_t n_s   + \frac{1}{B} \left \{ \psi_s, n_s \right \}
  =  n_s \kappa \partial_y \psi_s
  + \tau_s \kappa \partial_y n_s
  -\nu \nabla_{\perp}^4 n_s. \label{eq:density}
\end{equation}
The electric plasma potential $\phi({\bf x},t)$ appears in the full-f
gyrofluid potential
\begin{equation}
    \psi_s  =  \Gamma_s \phi - \frac{1}{2} \mu_s \left(
    \frac{\nabla_{\perp} \phi}{B} \right)^{2}
\end{equation}
The electric potential $\phi$ is determined through the quasi-neutral non-linear
polarization equation in full-f long wavelength form:

\begin{equation}
  \sum_s \left[  \nabla \cdot \left( \frac{\mu_s n_s}{Z_s B^2}
    \nabla_{\perp} \phi \right)  \right] =  - \sum_s \left[ Z_s \Gamma_s n_s  \right]
  \label{eq:polarization}
\end{equation}

For a non-localized impurity species, which features a constant static
background density $n_{z0}$ and contributes to the overall plasma quasi
neutrality through the equality $Z_i e n_{i0} + Z_j e n_{0z} - e n_{e0}$ = 0,
the polarization equation can also be expressed by defining the density ratios
$a_s = Z_s n_{s0} / n_{e0}$. Then ${\bar \epsilon_s} = \sum_s a_s \mu_s n_s / B^2$,
and $Q =  - \sum_s [ a_s \Gamma_s n_s ]$.
Both, the general (localized) and the background impurity forms, will be used
in the following sections.

\subsection{Initialization}

\subsubsection{Seeded blob on constant background}
We here assume one main ion and one impurity species.
We assume constant background densities $a_j < a_i$ for all species and initialize a blob
with same relative impurity concentration.

Initialization of $n$ is a Gaussian
\begin{align}
    n(x,y) = 1 + A\exp\left( -\frac{(x-X)^2 + (y-Y)^2}{2\sigma^2}\right)
    \label{}
\end{align}
where $X = p_x l_x$ and $Y=p_yl_y$ are the initial centre of mass position coordinates, $A$ is the amplitude and $\sigma$ the
radius of the blob.
We initialize
\begin{align}
    N = \Gamma_1^{-1} n \quad \phi = 0 \\
    \rho = \phi = 0
    \label{}
\end{align}
\subsection{Diagnostics}
\begin{align}
    M(t) = \int n-1 \\
    \Lambda_n = \nu \int \Delta n  \\
    ...
    \label{}
\end{align}
\section{Numerical methods}
discontinuous Galerkin on structured grid
\rowcolors{2}{gray!25}{white} %%% Use this line in front of longtable
\begin{longtable}{ll>{\RaggedRight}p{7cm}}
\toprule
\rowcolor{gray!50}\textbf{Term} &  \textbf{Method} & \textbf{Description}  \\ \midrule
coordinate system & Cartesian 2D & equidistant discretization of $[0,l_x] \times [0,l_y]$, equal number of Gaussian nodes in x and y \\
matrix inversions & conjugate gradient & Use previous two solutions to extrapolate initial guess and $1/\chi$ as preconditioner \\
\ExB advection & Arakawa & s.a. \cite{Einkemmer2014} \\
curvature terms & direct & flux conserving \\
time &  Karniadakis multistep & $3rd$ order explicit, diffusion $2nd$ order implicit \\
\bottomrule
\end{longtable}

\section{Compilation and useage}
There are two programs toeflR.cu and toefl\_hpc.cu . Compilation with
\begin{verbatim}
make <toeflR toefl_hpc toefl_mpi> device = <omp gpu>
\end{verbatim}
Run with
\begin{verbatim}
path/to/feltor/src/toefl/toeflR input.json
path/to/feltor/src/toefl/toefl_hpc input.json output.nc
echo np_x np_y | mpirun -n np_x*np_y path/to/feltor/src/toefl/toefl_mpi\
    input.json output.nc
\end{verbatim}
All programs write performance informations to std::cout.
The first is for shared memory systems (OpenMP/GPU) and opens a terminal window with life simulation results.
 The
second can be compiled for both shared and distributed memory systems and uses serial netcdf in both cases
to write results to a file.
For distributed
memory systems (MPI+OpenMP/GPU) the program expects the distribution of processes in the
x and y directions as command line input parameters.

\subsection{Input file structure}
Input file format: \href{https://en.wikipedia.org/wiki/JSON}{json}
\begin{minted}[texcomments]{js}
{
    "n" : 3,  // \# Gaussian nodes in x and y
    "Nx" : 100, // \# grid points in x
    "Ny" : 100,  // \# grid points in y
    "dt" : 3.0,  // time step in units of $c_s/\rho_s$
    "n_out"  : 3,  // \# Gaussian nodes in x and y in output
    "Nx_out" : 100, // \# grid points in x in output fields
    "Ny_out" : 100, // \# grid points in y in output fields
    "itstp"  : 2,   // steps between outputs
    "maxout" : 100, // \# outputs excluding first
    "eps_pol"   : 1e-6    // accuracy of polarisation solver
    "eps_gamma" : 1e-7    // accuracy of $\Gamma_1$ (only in gyrofluid model)
    "eps_time"  : 1e-10,   // accuracy of implicit time-stepper
    "curvature"  : 0.00015,// magnetic curvature $\kappa$
    "tau"        : 1.0,      // $\tau = T_i/T_e$ (only in gyrofluid models)
    "nu_perp"   : 5e-3,    // pependicular viscosity $\nu$
    "amplitude"  : 1.0,    // amplitude $A$ of the blob
    "sigma"      : 10.0,   // blob radius $\sigma$
    "posX"       : 0.3,    // blob x-position in units of $l_x$, i.e. $X =p_x l_x$
    "posY"       : 0.5,    // blob y-position in units of $l_y$, i.e. $Y =p_y l_y$
    "lx"         : 200.0,  // $l_x$
    "ly"         : 200.0,  // $l_y$
    "friction"   :  0.0,   // friction coefficient $\eta$ in gravity model
    "bc_x"   : "DIR",     // boundary condition in x (one of PER, DIR, NEU, DIR\_NEU or NEU\_DIR)
    "bc_y"   : "PER",      // boundary condition in y (one of PER, DIR, NEU, DIR\_NEU or NEU\_DIR)
    "equations"  : "global", // "local", "global", "gravity\_local", "gravity\_global", "drift\_global"
    "boussinesq" : false,    // boussinesq approximation in global models true or false
}
\end{minted}

The default value is taken if the value name is not found in the input file. If there is no default and
the value is not found,
the program exits with an error message.

\subsection{Structure of output file}
Output file format: netcdf-4/hdf5
%
%Name | Type | Dimensionality | Description
%---|---|---|---|
\begin{longtable}{lll>{\RaggedRight}p{7cm}}
\toprule
\rowcolor{gray!50}\textbf{Name} &  \textbf{Type} & \textbf{Dimension} & \textbf{Description}  \\ \midrule
inputfile  &             text attribute & 1 & verbose input file as a string \\
energy\_time             & Dataset & 1 & timesteps at which 1d variables are written \\
time                     & Dataset & 1 & time at which fields are written \\
x                        & Dataset & 1 & x-coordinate  \\
y                        & Dataset & 1 & y-coordinate \\
electrons                & Dataset & 3 (time, y, x) & electon density $n$ \\
ions                     & Dataset & 3 (time, y, x) & ion density $N$ or vorticity density $\rho$  \\
potential                & Dataset & 3 (time, y, x) & electric potential $\phi$  \\
vorticity                & Dataset & 3 (time, y, x) & Laplacian of potential $\nabla^2\phi$  \\
dEdt                     & Dataset & 1 (energy\_time) & change of energy per time  \\
dissipation              & Dataset & 1 (energy\_time) & diffusion integrals  \\
energy                   & Dataset & 1 (energy\_time) & total energy integral  \\
mass                     & Dataset & 1 (energy\_time) & mass integral   \\
\bottomrule
\end{longtable}
\section{Diagnostics toeflRdiag.cu}
There only is a shared memory version available
\begin{verbatim}
cd path/to/feltor/diag
make toeflRdiag
path/to/feltor/diag/toeflRdiag input.nc output.nc
\end{verbatim}

Input file format: netcdf-4/hdf5
%
%Name | Type | Dimensionality | Description
%---|---|---|---|
\begin{longtable}{lll>{\RaggedRight}p{7cm}}
\toprule
\rowcolor{gray!50}\textbf{Name} &  \textbf{Type} & \textbf{Dimension} & \textbf{Description}  \\ \midrule
inputfile  &             text attribute & 1 & verbose input file as a string \\
electrons                & Dataset & 3 & electon density (time, y, x) \\
ions                     & Dataset & 3 & ion density (time, y, x) \\
potential                & Dataset & 3 & electric potential (time, y, x) \\
\bottomrule
\end{longtable}

Output file format: netcdf-4/hdf5
%
%Name | Type | Dimensionality | Description
%---|---|---|---|
\begin{longtable}{lll>{\RaggedRight}p{7cm}}
\toprule
\rowcolor{gray!50}\textbf{Name} &  \textbf{Type} & \textbf{Dimension} & \textbf{Description}  \\ \midrule
 inputfile & text attribute & 1 & copy of inputfile attribute of the input file (the json string of the simulation input file) \\
 time & Dataset & 1 & the time steps at which variables are written \\
 posX & Dataset & 1 (time) & centre of mass (COM) position x-coordinate \\
 posY & Dataset & 1 (time) &COM y-position \\
 velX & Dataset & 1 (time)& COM x-velocity \\
 velY & Dataset & 1 (time)& COM y-velocity \\
 accX & Dataset & 1 (time)& COM x-acceleration \\
 accY & Dataset & 1 (time)& COM y-acceleration \\
 velCOM & Dataset & 1 (time)&absolute value of the COM velocity \\
 posXmax& Dataset & 1 (time)&maximum amplitude x-position \\
 posYmax& Dataset & 1 (time)&maximum amplitude y-position \\
 velXmax& Dataset & 1 (time)&maximum amplitude x-velocity \\
 velYmax& Dataset & 1 (time)&maximum amplitude y-velocity \\
 maxamp & Dataset & 1 (time)&value of the maximum amplitude  \\
  compactness\_ne& Dataset & 1 (time) &compactness of the density field \\
 Ue& Dataset&  1 (time) &entropy electrons \\
 Ui &Dataset& 1 (time) & entropy ions \\
 Uphi& Dataset& 1 (time) &  exb energy \\
 mass& Dataset & 1 (time) & mass of the blob without background \\
\bottomrule
\end{longtable}


%..................................................................
\bibliography{../common/references}
%..................................................................


\end{document}
