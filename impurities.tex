%%%%%%%%%%%%%%%%%%%%%definitions%%%%%%%%%%%%%%%%%%%%%%%%%%%%%%%%%%%%%%%

\input{../feltor/src/common/header.tex}
\input{../feltor/src/common/newcommands.tex}
\usepackage{minted}
\newcommand{\ns}{\ensuremath{{N_s} }}
\newcommand{\physs}{\ensuremath{{n_s} }}

%%%%%%%%%%%%%%%%%%%%%%%%%%%%%DOCUMENT%%%%%%%%%%%%%%%%%%%%%%%%%%%%%%%%%%%%%%%
\begin{document}

\title{The impurities project}
\author{ M.~Wiesenberger and M.~Held and E.~Reiter}
\maketitle

\begin{abstract}
  This is a program for 2d isothermal blob simulations with an arbitrary number of species
\end{abstract}
\tableofcontents

\section{Compilation and useage}
The program impurities.cpp can be compiled three ways with
\begin{verbatim}
make <impurities impurities_hpc impurities_mpi> device = <cpu omp gpu>
\end{verbatim}
Run with
\begin{verbatim}
path/to/impurities/impurities input.json
path/to/impurities/impurities_hpc input.json output.nc
echo np_x np_y | mpirun -n np_x*np_y path/to/impurities/impurities_mpi\
    input.json output.nc
\end{verbatim}
All programs write performance informations to std::cout.
The first is for shared memory systems (CPU/OpenMP/GPU) and opens a terminal window with life simulation results.
 The
second is shared memory systems and uses serial netcdf
to write results to a file.
For distributed
memory systems (MPI+CPU/OpenMP/GPU) the program expects the distribution of processes in the
x and y directions as command line input parameters. Also there serial netcdf is used.

\section{Model}

The dimensionless equations for the evolution of the gyro-center
densities $\ns({\bf x},t)$ in two-dimensional Cartesian coordinates $x,y$ for each species are:

\begin{align} \label{eq:density}
    \partial_t \ns  &=  - \frac{1}{B} \left \{ \psi_s, \ns \right \}
  +  \ns \kappa \partial_y \psi_s
  + \tau_s \kappa \partial_y \ns
  -\nu_{s,\perp} \nabla_{\perp}^4 \ns \\
  &= -(\vec{ v_{E,s}} + \vec{ v_{\kappa,s}} )\cdot \ns  - \ns \nabla\cdot \vec {v_E} -\nu_{s,\perp} \Delta_\perp^2 \ns
\end{align}
where the Arakawa bracket represents the advection by the \ExB velocity
\begin{align}
    \frac{1}{B} \left \{ \psi_s, \ns \right \} = \vec v_{E,s} \cdot \nabla \ns \\
v_x = -\frac{1}{B} \partial_y \psi_s \quad v_y = \frac{1}{B}\partial_x \psi_s
\end{align}
The magnetic field is given by
\begin{align}
    \vec B=  B(x) \hat e_z \quad
    B^{-1}(x)= 1+\kappa x
\end{align}
from which the curvature operators $\vec K = \frac{1}{B^2}\frac{\partial B}{\partial x}\hat e_y = -\partial_x B^{-1}\hat e_y = -\kappa \hat e_y$.
and
\begin{align}
\vec v_{\kappa,s} = \tau \vec K = -\kappa\tau_s \hat e_y
\end{align}

The electric plasma potential $\phi({\bf x},t)$ appears in the full-f
gyrofluid potential
\begin{equation}
    \psi_s  =  \Gamma_s \phi - \frac{1}{2} \mu_s \left(
    \frac{\nabla_{\perp} \phi}{B} \right)^{2}
\end{equation}
The electric potential $\phi$ is determined through the quasi-neutral non-linear
polarization equation in full-f long wavelength form:
\begin{equation}
    -\sum_s \left[\nabla \cdot \left(  \left(\eps_D +  \frac{a_s \mu_s \ns}{ B^2}\right)
    \nabla_{\perp} \phi \right)  \right] =  \sum_s a_s \Gamma_s \ns 
  \label{eq:polarization}
\end{equation}
where $\eps_D$ is the (optional) Debye parameter (important for electron-positron plasmas)
and the gyro-averaging operator
\begin{align}
    \Gamma_s  =  \left( 1- \frac{1}{2} \tau_s \mu_s \nabla_{\perp}^2 \right)^{-1}
\end{align}
is expressed in Pad\'e approximation.
We note the physical particle densities
\begin{align}
    \physs = \Gamma_{1s}\ns + \nabla\cdot\left(\frac{ \mu_s \ns}{ B^2}
    \nabla_{\perp} \phi \right)
\end{align}
We have
\begin{align}
    a_s = \frac{Z_s n_{s0}}{n_{e0}}  \quad \mu_s = \frac{m_s}{Z_s m_i} \quad \tau_s = \frac{T_s}{Z_s T_e}
\end{align}

The overall plasma quasi-neutrality reads
\begin{align}
    \sum_s a_s = 0
\end{align}
%%%%%%%%%%%%%%%%%%%%%%%%%%%%%%%%%%%%%%%
\section{Invariants}
\subsection{Mass}
The particles density conservation reads
\begin{align} \label{eq:mass_theorem}
  \frac{\partial}{\partial t} \mathcal M_s
  + \nc\vec{ j_{n,s}}
  =  \Lambda_{\ns}
\end{align}
with
\begin{align}
    \mathcal M &= \ns \\
    \vec j_{n,s} &= \ns \vec v_E + \vec v_\kappa \\
     \Lambda_\ns &= -\nu_\perp \Delta_\perp^2 \ns
\end{align}
With vanishing flux on the boundaries and zero viscosity we have
\begin{align}
    \frac{\partial}{\partial t} \int \dV \mathcal M_s = 0
\end{align}

\subsection{Energy}
The energy theorem reads
\begin{align} \label{eq:energy_theorem}
  \frac{\partial}{\partial t} \mathcal E
  + \nc\vec{ j_{\mathcal E}}
  =  \Lambda_{\mathcal E}
\end{align}
The inherent energy density of our system is:
\begin{align}
 \mathcal{E} := &
 \sum_s a_s \tau_s \ns \ln{(\ns)} + \frac{1}{2} a_s \mu_s \ns u_E^2
\end{align}
The energy current density and diffusion are
\begin{align}
  \vec j_{\mathcal E} =& \sum_s a_s\left[
  \left(\tau_s \ln \ns + \psi_s \right)\ns\left(
  \vec u_E + \vec u_\kappa \right) \right]
  , \\
    \Lambda_\mathcal{E} :=  &+ \left( \tau_s ( 1+\ln \ns) + \psi_s\right)(-\nu_\perp \Delta_\perp^2 \ns)
    \label{eq:energy_diffusion}
\end{align}
where in the energy flux $\vec j_{\mathcal E}$
we neglect terms  containing time derivatives
of the eletric potentials and we sum over all species.

With our choice of boundary conditions the energy flux
vanishes on the boundary.
In the absence of artificial viscosity the volume integrated
 energy density is thus an exact invariant of the system.
\begin{align}
    \frac{\partial}{\partial t} \int \dV \mathcal E = 0
\end{align}
%%%%%%%%%%%%%%%%%%%%%
\section{Blob related quantities}
For the blob quantities it is important that we use the physical densities $\physs$.
\begin{tcolorbox}[title=Note]
    These quantities can be easily computed in post-processing in e.g. a python script.
\end{tcolorbox}

\subsection{Center of mass}
The first quantity of interest is the Center-of-mass position
\begin{align}
    M :=& \int (\physs -1 ) \dV \\
    X :=& \frac{1}{M} \int x(\physs - 1) \dV \\
    Y :=& \frac{1}{M} \int y(\physs - 1) \dV
\end{align}
\subsection{Blob compactness}
The blobs ability to retain its initial (Gaussian) shape is quantified by the  blob compactness
\begin{align}
     I_c(t) &:= \frac{\int dA (\physs(\vec{x},t)-1) h(\vec{x},t)}{\int dA
(\physs(\vec{x},0)-1) h(\vec{x},0)}
\end{align}
Here, we introduced the heaviside function
\begin{align}
     h(\vec{x},t) &:= \begin{cases}
          1,
        &\ \text{if} \hspace{2mm}||\vec{x} - \vec{X}_{max}||^2 < \sigma^2 \\
0,  &\ \text{else}
           \end{cases} \nonumber
\end{align}
and the position of the maximum density \( \vec{X}_{max}(t)\).


\section{The Input file} \label{sec:input_file}
Input file format: \href{https://en.wikipedia.org/wiki/JSON}{json}
%%%%%%%%%%%%%%%%%%%%%%%%%
\subsection{Spatial grid} \label{sec:spatial}
The spatial grid is an equidistant discontinuous Galerkin discretization of the
2D Cartesian product-space
$[ x_0, x_1]\times [y_0, y_1]$,
We use an equal number of Gaussian nodes in $x$ and $y$.
\begin{minted}[texcomments]{js}
"grid" :
{
    "n"  :  3, // The number of Gaussian nodes in x and y (3 is a good value)
    "Nx"  : 48, // Number of cells in x
    "Ny"  : 48, // Number of cells in y
    "x"  : [0, 200], // $[x_0, x_1]$
    "y"  : [0, 200], // $[y_0, y_1]$
}
\end{minted}
\subsection{Elliptic solvers}
We discretize all elliptic operators with a local dG method (LDG).  In order to
solve the elliptic equations we chose a multigrid scheme (nested iterations in
combination with conjugate gradient solves on each plane). The accuaracies for
the polarization equation can be chosen for each stage separately, while for
the Helmholtz type equations (the gamma operators) only
one accuracy can be set (they typically are quite fast to solve):
\begin{minted}[texcomments]{js}
"elliptic":
{
    "stages"    : 3,  // Number of stages (3 is best in virtually all cases)
    // $2^{\text{stages-1}}$ has to evenly divide both $N_x$ and $N_y$
    "eps_pol"   : [1e-6,10,10],
    // The first number is the tolerance for residual of the inversion of
    // polarisation equation. The second number is a multiplicative
    // factor for the accuracy on the second grid in a multigrid scheme, the
    // third for the third grid and so on:
    // $\eps_0 = \eps_{pol,0}$, $\eps_i = \eps_{pol,i} \eps_{pol,0}$  for $i>1$.
    // Tuning those factors is a major performance tuning oppourtunity!!
    // For saturated turbulence the suggested values are [1e-6, 2000, 100].
    "eps_gamma" : [1e-10,1,1] // Accuracy requirement of Gamma operator
    "direction" : "forward", // Direction of the Laplacian: forward or centered
}
\end{minted}
\begin{tcolorbox}[title=Note]
    We use solutions from previous time steps to extrapolate an initial guess
    and use a diagonal preconditioner
\end{tcolorbox}
\subsection{Magnetic field}
The parameter $\kappa$ is the curvature given by the input parameter
\begin{minted}[texcomments]{js}
{
    "curvature" : 0.000457 // $\kappa$
}
\end{minted}
\subsection{Electric potential}
\begin{minted}[texcomments]{js}
"potential" :
{
    "epsilon_D" : 0.,
    "bc" : ["DIR", "PER"]
},
\end{minted}

\subsection{Species Initialization}
Each species is defined by an entry in the species list in the input file:
\begin{minted}[texcomments]{js}
"species":
[
    {
        "name" : "electrons",
        "mu" : 0,
        "tau" : -1,
        "a" :  -1,
        "nu_perp" : 1e-6,
        "init":
        {
            // \ldots
        },
        bc :["DIR", "PER"]
    },
    {
        "name" : "ions",
        "mu" : 1,
        "tau" : 0,
        "a" :  0.9,
        "nu_perp" : 1e-6,
        "init":
        {
            // \ldots
        },
        bc :["DIR", "PER"]
    },
    // \ldots extend here
],
\end{minted}
The number of species is implicitly given by the length of the list.
\begin{tcolorbox}[title=Note]
    The program will terminate if $\sum_s a_s \neq 0$. The species names chosen in the input file reflect in the names in the output file
\end{tcolorbox}

For the "init" field various schemes can be chosen
\subsubsection{Constant background}
A constant background $\ns = 1$ can be chosen via
\begin{minted}[texcomments]{js}
{
    "type" : "const"
}
\end{minted}
\subsubsection{Seeded blob}
Initialization of $\ns$ as a Gaussian
\begin{align} \label{eq:profile_blob}
    \bar \ns(x,y) &= 1 + A\exp\left( -\frac{(x-X)^2 + (y-Y)^2}{2\sigma^2}\right) \\
    \ns(x,y) &= \bar \ns(x,y) \text{ no FLR correction} \\
    \ns(x,y) &= \Gamma_{1s}\bar \ns(x,y) \text{ FLR correction only for } \\
    \ns(x,y) &= \Gamma_{1s}^{-1}\bar \ns(x,y) \text{ inverse FLR correction} \\
\end{align}
where $X = x_0 + p_x (x_1 - x_0)$ and $Y= y_0 + p_y( y_1 - y_0)$ are the initial centre of mass position coordinates, $A$ is the amplitude and $\sigma$ the
radius of the blob.
\begin{minted}[texcomments]{js}
{
    "type" : "blob",
    "amplitude": 1.0, // $A$ in Eq.\eqref{eq:profile_blob}
    "posX" : 0, // $p_x$ in Eq.\eqref{eq:profile_blob}
    "posY" : 0, // $p_y$ in Eq.\eqref{eq:profile_blob}
    "sigma" : 5.0, // $\sigma$ in Eq.\eqref{eq:profile_blob}
    "flr" : "none", //only needed for $\tau_s \mu_s\neq 0$
    "flr" : "gamma", //only needed for $\tau_s \mu_s\neq 0$
    "flr" : "gamma_inv", //only needed for $\tau_s \mu_s\neq 0$
}
\end{minted}
\subsubsection{Wall}
Initialization of $\ns$ as a Gaussian only in the x-direction
\begin{align} \label{eq:profile_wall}
    \bar \ns(x,y) &= 1 + A\exp\left( -\frac{(x-X)^2 }{2\sigma^2}\right) \\
    \ns(x,y) &= \bar \ns(x,y) \text{ no FLR correction} \\
    \ns(x,y) &= \Gamma_{1s}\bar \ns(x,y) \text{ FLR correction} \\
    \ns(x,y) &= \Gamma_{1s}^{-1}\bar \ns(x,y) \text{ inverse FLR correction} \\
\end{align}
where $X = x_0 + p_x (x_1 - x_0)$ is the initial centre of mass position coordinates, $A$ is the amplitude and $\sigma$ the
radius of the wall.
\begin{minted}[texcomments]{js}
{
    "type" : "wall",
    "amplitude": 1.0, // $A$ in Eq.\eqref{eq:profile_wall}
    "posX" : 0, // $p_x$ in Eq.\eqref{eq:profile_wall}
    "sigma" : 5.0, // $\sigma$ in Eq.\eqref{eq:profile_wall}
    "flr" : "none", //only needed for $\tau_s \mu_s\neq 0$
    "flr" : "gamma", //only needed for $\tau_s \mu_s\neq 0$
    "flr" : "gamma_inv", //only needed for $\tau_s \mu_s\neq 0$
}
\end{minted}

\subsubsection{ Set potential zero}
A (single) field can be left to be determined by the condition $\phi = 0$ i.e.
\begin{align}
    \sum_s a_s \Gamma_{1s} \ns = 0
\end{align}
This currently works only for mass-less species with $\mu_s =0$ (i.e. electrons)
\begin{align}
    n_e(x,y) = -\sum_{s\neq e} \frac{a_s}{a_e} \Gamma_{1s}\ns
\end{align}
\begin{minted}[texcomments]{js}
{
    "type" : "zero_potential"
}
\end{minted}
\subsection{Timestepper}
We use an adaptive explicit embedded Runge Kutta timestepper to advance the equations in time
\begin{minted}[texcomments]{js}
"timestepper":
{
    "tableau" : "Bogacki-Shampine-4-2-3",
    "rtol" : 1e-5,
    "atol" : 1e-6
}
\end{minted}

\section{Output}
Our program can either write results directly to screen using the glfw library
or write results to disc using netcdf.
This can be controlled via
\begin{minted}[texcomments]{js}
"output":
{
    // Use glfw to display results in a window while computing (requires to
    // compile with the glfw3 library)
    "type"  : "glfw"
    "itstp"  : 4, // The number of steps between refreshing field plots

    // Use netcdf to write results into a file (filename given on command line)
    // (see next section for information about what is written in there)
    "type"  : "netcdf"
    "tend"  : 4,    // The number of steps between outputs of 2d fields
    "maxout"  : 100, // The total number of field outputs
    "n"  : 3 , // The number of polynomial coefficients in the output file
    "Nx" : 48, // Number of cells in x in the output file
    "Ny" : 48  // Number of cells in y in the output file
}
\end{minted}
The number of points in the output file can be lower (or higher) than the number of
grid points used for the calculation. The points will be interpolated from the
computational grid.
\subsection{Netcdf file}
Output file format: netcdf-4/hdf5

\begin{longtable}{lll>{\RaggedRight}p{7cm}}
\toprule
\rowcolor{gray!50}\textbf{Name} &  \textbf{Type} & \textbf{Dimension} & \textbf{Description}  \\ \midrule
inputfile        & text attribute & 1 & verbose input file as a string \\
time             & Coord. Var. & 1 (time) & time at which fields are written \\
x                & Coord. Var. & 1 (x) & x-coordinate  \\
y                & Coord. Var. & 1 (y) & y-coordinate \\
xc               & Dataset & 2 (y,x) & Cartesian x-coordinate  \\
yc               & Dataset & 2 (y,x) & Cartesian y-coordinate \\
weights          & Dataset & 2 (y,x) & Gaussian integration weights \\
X                & Dataset & 3 (time, y, x) & 2d outputs \\
\bottomrule
\end{longtable}
The output fields X are determined in the file \texttt{impurities/diag.h}.

%..................................................................
\bibliography{../common/references}
%..................................................................

\end{document}
